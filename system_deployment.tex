\chapter{Prototype System Deployment}
\label{chp:sys_deploy}

\noindent In this Chapter, there will be three main topics discussed because of deploying the prototype system to production usage.

\section{TURN Server Deployment}

\par During the development of the prototype system, the test based on XMS media server is not stable at the beginning. There is one way audio issue happens in the prototype system when \gls{webrtc} client init a outbound call to a \gls{sip} client(mobile phone). Since it is working fine when the outbound \gls{sip} client call into the \gls{webrtc} client, after tracing the network log from the XMS media server, the problem is the \gls{ice} candidate information got from the original \gls{stun} server can not punch the whole on the firewall of the XMS media server. It is normal to replace the \gls{stun} server as \gls{turn} server to solve this problem because if the \gls{stun} server way is blocked during the media stream exchange from two end point, it will switch to \gls{turn} server way to exchange the media stream through the \gls{turn} server to rely all the media traffic.

\par Moreover, the \gls{turn} server solution will work well in the different corporation networks scenario since there will be highly restrict corporation firewall in front of the corporation network. Then \gls{turn} server can rely all the media stream to establish the peer to peer connection. After testing prototype system against \gls{turn} server instead of \gls{stun} server provided by Google (shown at line 3 in Code Snippet \ref{code:create_peer_connection}), the one way audio issue is solved and two end client in different corporation network scenario works fine as well.

\par To set up \gls{turn} server in the production of prototype system, the prototype system uses \gls{aws} \gls{ec2}\footnote{Amazon Elastic Compute Cloud (EC2) is a central part of Amazon.com's cloud computing platform, Amazon Web Services (AWS). EC2 allows users to rent virtual computers on which to run their own computer applications. EC2 allows scalable deployment of applications by providing a Web service through which a user can boot an Amazon Machine Image to create a virtual machine, which Amazon calls an "instance", containing any software desired. A user can create, launch, and terminate server instances as needed, paying by the hour for active servers, hence the term "elastic". EC2 provides users with control over the geographical location of instances that allows for latency optimization and high levels of redundancy.\cite{wiki:ec2}}, \gls{ip} address: 54.187.157.224. There is a free open source implementation of \gls{turn} and \gls{stun} server maintained by Google. It provides the \gls{aws} \gls{ec2} hosting image, then it is only necessary to configure the \gls{aws} \gls{ec2} virtual instance to open the necessary ports for the \gls{turn} server usage. It is shown in the following list:
\begin{itemize}[topsep=-1em,parsep=0em,itemsep=0em]
 \item TCP 443
 \item TCP 3478-3479
 \item TCP 32355-65535
 \item UDP 3478-3479
 \item UDP 32355-65535
\end{itemize}

\par Moreover, the \gls{turn} server can either use a flat file or a \gls{sql} database for configuration and user information. In the prototype system, the \gls{turn} server on \gls{aws} \gls{ec2} will use a flat file for configuration and user information. It is edited in "/usr/local/etc/turnuserdb.conf" by adding an entry on its own line: "my\_username:my\_password".\cite{dialogic:turn} Other configurations need to be completed by following the README file under the hosting instance image directory on \gls{aws} \gls{ec2}. There are several paramters need to be set in "/etc/turnserver.conf" on \gls{turn} server.

\par Besides establishment for \gls{turn} server, there are some changes need to be done on the client application as well in order to use this \gls{turn} server to fetch the useful \gls{ice} candidate information during the peer to peer connection. Compare to the Code Snippet \ref{code:create_peer_connection} with original Google \gls{stun} server address, in Code Snippet \ref{code:client_turn_server}, prototype \gls{turn} server is set as \textit{iceServer}. The \textit{iceTransports} field is the parameter to force client to use \gls{turn} server but it is only purposed by Google Chrome, it is not standard and it is not implemented in Google Chrome yet.

\begin{lstlisting}[caption={Using TURN Server on WebRTC Client},label={code:client_turn_server}]
if (location.hostname != "localhost") {
  			
				pc_config = 
				{
					'iceServers': [{
						'urls': 'turn:54.187.157.224',
						'username': 'my_username',
						'credential': 'my_password'
					}],
					'iceTransports': 'relay'
				};
			}
\end{lstlisting}


\section{Application Server Deployment}

\par Because the prototype application server is implemented in Node.js, there is no restrict requirements for the operation system platform to deploy the application server if the operation system can install Node.js library and can run V8 JavaScript Engine\footnote{The V8 JavaScript Engine is an open source JavaScript engine developed by Google for the Google Chrome web browser.V8 compiles JavaScript to native machine code (IA-32, x86-64, ARM, or MIPS ISAs) before executing it, instead of more traditional techniques such as interpreting bytecode or compiling whole program to machine code and executing it from a filesystem. \cite{wiki:v8}}.

\par It also needs to open the \textit{5060} port to support \gls{udp} for \gls{sip} stack usage. Then it just need to use \textit{node server.js} command to host the application server for production.

\section{XMS Server Deployment}

\par The XMS media server is host on a stand alone machine during the development. For deployment reason, it is necessary to map the internal \gls{ip} address of XMS media server to a public \gls{ip} address. And it is important to open the necessary port for the XMS media usage. According to the documentation of Dialogic PowerMedia XMS 
Installation and Configuration Guide\cite{dialogic:xms_install},the default PowerMedia XMS configuration uses the following ports:
\begin{itemize}[topsep=-1em,parsep=0em,itemsep=0em]
 \item \gls{tcp}: 22, 80, 81, 443, 5060, 1080, 15001 
 \item \gls{udp}: 5060, 49152-53152, 57344-57840 
\end{itemize}

\par Because the application server and XMS media server in the prototype system are host in the corporation network, it only opens necessary port to the public network. During the exchange \gls{ice} candidate information for client, the XMS \gls{ip} address will be the internal \gls{ip} address by the rule of the corporation network. It is necessary to change the internal \gls{ip} address into public \gls{ip} address before pushing the \gls{ice} candidate information back to the end point client. This process is implemented at line 37 in Code Snippet \ref{code:xms}, it simply just replaces the internal \gls{ip} address as public \gls{ip} address of XMS media server in the \gls{sdp} content string.

