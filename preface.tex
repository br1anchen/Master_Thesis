\renewcommand{\abstractname}{Preface}
\begin{abstract}
\noindent \gls{webrtc} is quite popular topic in the web development filed since the massive usage and development of \gls{html5} web application on the internet. The initial purpose of this web \gls{api} is to provide the browser client the ability to create real-time conversation between each other. After many \gls{webrtc} based application come out the market, it is quite normal to think about how to integrate these kind of web application with the current legacy telephony network as the next big step for this technology. The requirement of this process is not only from the traditional telephony operator but also the normal end-users. The approach to achieve this goal is the man purpose of this thesis.
\par Research about current \gls{webrtc} technology usage and development of a \gls{webrtc} prototype system are the two main parts of this thesis. The prototype system is implemented by regarding to the research of \gls{webrtc} integrated with legacy telephony network.
\par Current status of \gls{webrtc} technology, \gls{webrtc} business use cases, analysis of different possible \gls{webrtc} implement solutions and \gls{webrtc} system architecture will be covered in this thesis. Some research regarding with the development of \gls{webrtc} prototype system will be covered in this thesis as well.
\par The prototype described in this thesis is implemented to cooperate with existing legacy \gls{voip} network services through \gls{sip} server and \gls{pbx} service. It will provide most of essential functions which are included in the legacy telephony business, besides other communication functions used on web. Moreover, some analysis and discussion about the feedback of the prototype will be covered in this thesis.
\par The prototype will be implemented in programming language Javascript for both client font-end and server back-end by using the AngularJs framework and Nodejs framework mainly. The approach and reason to choose these framework and programming language will be expounded in the later chapter in this thesis.
\end{abstract}