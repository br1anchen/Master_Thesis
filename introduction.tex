\chapter{Introduction}
\label{chp:intro}

\noindent To make an unified communication solution with \gls{webrtc}, integrating \gls{webrtc} technology with traditional telephony network is the main goal of it. The term, unified communication, in this thesis means the unified solution for real time communication on the internet and on the traditional telephony network.

\section{Background and Motivation}

\par As the development of smart mobile phone industry, there are more and more people connected to the internet through smart phones. The real time communication demands is from the traditional telephony network to \gls{ip} network. There are many client applications provide real time communication service through the internet. There are two main different categories for these real time communication solution. One kind of application is like Google Hangout, it provides user a real time communication channel on the internet and require user client both using browser to communicate with each other. The other kind of application is like Skype\footnote{Skype is a freemium voice-over-IP service and instant messaging client, currently developed by the Microsoft Skype Division. The name was derived from "sky" and "peer".\cite{wiki:skype}}, it provides \gls{voip} service and let different client users(browser and physical phone) to communicate with each other.

\par However, the problem of the second category application is that users have to install some application client and request for some application credential to use the service. There are already many different applications installed on user's smart phone and desktop computer. It is hard for user to remember another application credential and install one more application for just calling.

\par The motivation of this thesis is to provide an unified communication solution for user easily to have real time conversation with the other user either on mobile phone or desktop computer. The unified communication solution should not demand user to install any client software or plugins and not ask user to remember another new credential information either.

\par The approach for that would be a web application service using user telephone number as credential and provide the user call any kind of other user no matter the other user is on his mobile phone or his computer through internet. This system will be a \gls{ott} solution integrated with \gls{webrtc} network and \gls{voip} network. The service can provide user a new real time communication way to reach other people in the world since every one is on the internet or on the phone nowadays.

\par The prototype system implemented in this thesis will provide rich multimedia real-time communication service with \gls{webrtc} network and \gls{sip} network. Some basic real-time communication application functions will be achieved, like calling mobile phone, having video conferencing, instance texting and 

\section{Method}

\noindent The main focus of this thesis is to research about how to make an unified communication system with \gls{webrtc} technology. The approach to this goal will be studies about \gls{webrtc} and its commercial usage and also the prototype implementation to verify the prototype system design and some ideas about the way to implement this unified communication system. The approach for this goal in the thesis is to implement the prototype system and demo tests to understand the different solutions and analysis their advantages and disadvantages.

\par There will be different case studies and implementation solution demo testing in the thesis. They are helping the research about the unified communication solution. Besides the understanding and analysis on the demo testing and case studies, the prototype system implementation will give more feedback and future approach to finalize the unified communication system.

\par Although there will software application development for prototype system in this thesis, the design about the application logic of the software development will not be included in this thesis.

\par The decision of the prototype system implementation method is based on the comparison of different implementation solutions in the Chapter \ref{chp:sys_design}. All the comparison of these implementation solutions are made based on the demo testing in these different solutions.

\par After the prototype system implementation, the performance is judged by the student and some other pilot testers. Because the prototype system in this thesis is to prove the implementation solution and system design for the unified communication service with \gls{webrtc}, the user experience and prototype system performance is not the main focus in this thesis. Although some analysis and discussion will cover these issues, they will not take big account of this thesis.

\par There will be discussion about the future work in this thesis based on the prototype system implementation. It is based on \gls{webrtc} case studies and feedback of the prototype system. It will help other researcher in the same field to have some reference on the potential and direction of the unified communication service with \gls{webrtc}.

\section{Thesis Structure}

\noindent There are five chapters about the process of creating an unified communication service with \gls{webrtc} technology in this thesis.

\par Chapter \ref{chp:pre_study} covers basic studies about \gls{webrtc} and \gls{sip}, these two technology. The reason to discuss \gls{sip} network is because the \gls{sip} signaling protocol is the most widely used \gls{voip} protocol in all the kind of real time communication services. And also the target \gls{voip} network \gls{pbx} in this thesis is \gls{sip} supported \gls{pbx}. In this chapter will also cover the basic working scenario of the prototype system based on the \gls{webrtc} usage example of the commercial products.

\par Chapter \ref{chp:sys_design} covers different solutions for the prototype system. They are implemented and tested in some demo tests. After comparing these demo tests, some choices will be made for the implementing process of the prototype system.

\par Chapter \ref{chp:sys_imp} covers some details about the key factor in the prototype system. There are some explanation and analysis about the way how the prototype implementing. The reason for this chapter is to support the discussion of chapter \ref{chp:sys_design} and also give more information about the prototype system functionalities.

\par Chapter \ref{chp:sys_deploy} covers the process to deploy the prototype system to make it working. Since the prototype system is targeting to telephony network and \gls{ip} network, it is necessary to deploy the prototype system and test it in the real working scenario not only the testing environment. In most of the case, the deployment of this kind of real time communication service will cause some trouble for the system itself which needs to be concerned in the development.

\par Chapter \ref{chp:future_work} covers more discussion about the future work of the prototype system according to the feedback and experience of the prototype system. Some discussion will be addressed against with some points in the Chapter \ref{chp:sys_design} as well.

