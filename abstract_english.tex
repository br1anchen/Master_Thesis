\pagestyle{empty}

\begin{abstract}

\noindent During the development of traditional telephony echo-systems, the cost of maintenaning traditional telephony network is getting higher and higher but the number of customer does not grow rapidly any more since almost every one has a phone to access the traditional telephony network. \gls{webrtc} is an \gls{api} definition drafted by the \gls{w3c} that supports browser-to-browser applications for voice calling, video chat, and \gls{p2p} file sharing without plugins.\cite{wiki:webRTC} \gls{webrtc}, along with other advances in \gls{html5} browsers, has the potential to revolutionize the way we all communicate, in both personal and business spheres.\cite{inbook:rtc-preface}

\par Research about current \gls{webrtc} technology usage and implementation of a \gls{webrtc} prototype system are the two main parts of this thesis. The prototype system is implemented based on the research about \gls{webrtc} integrated with legacy telephony network.

\par This thesis will cover the research about how to apply \gls{webrtc} technology with existing legacy \gls{voip} network. And one prototype system to archive the unified communication solution with \gls{webrtc} will be introduced in this thesis.

\par The prototype described in this thesis is implemented to cooperate with existing legacy \gls{voip} network services through \gls{sip} server and \gls{pbx}\footnote{Users of the PBX share a certain number of outside lines for making telephone calls external to the PBX.\cite{webopedia:pbx}} service. It will provide most of essential functions which are included in the legacy telephony business, besides other communication functions already used on web . Moreover, some analysis and discussion about the feedback of the prototype will be covered in this thesis.

\par The prototype will be implemented in programming language Javascript for both client font-end and server back-end by using the AngularJs framework and Nodejs framework mainly. The approach and reason to choose these framework and programming language will be expounded in the later chapter in this thesis.

\noindent \textbf{Keywords} : \keywordnames
\end{abstract}