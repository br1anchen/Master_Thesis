\pagestyle{empty}

\begin{abstract}
\noindent During the development of traditional telephony echo-systems, the cost of maintenaning traditional telephony network is getting higher and higher but the number of customer does not grow rapidly any more since almost every one has a phone to access the traditional telephony network. \gls{webrtc} is an \gls{api} definition drafted by the \gls{w3c} that supports browser-to-browser applications for voice calling, video chat, and \gls{p2p} file sharing without plugins.\cite{wiki:webRTC} “This technology, along with other advances in \gls{html5} browsers, has the potential to revolutionize the way we all communicate, in both personal and business spheres.”\cite{inbook:rtc-preface}
\par As network operators aspect, \gls{webrtc} provides many opportunities to the future telecommunication business module. To the users who have already had mobile service, operator can offer \gls{webrtc} service with session-based charging to the existing service plans. Messaging \gls{api}s can augment \gls{webrtc} web application with \gls{rcs} and other messaging services which developers have already implemented. Furthermore, since \gls{webrtc} is a web based \gls{api}, then the implementation of \gls{qos} for \gls{webrtc} can provide assurance to users and prioritize services (enterprise, emergency, law enforcement, eHealth) that a WebRTC service will work as well as they need it to. \gls{webrtc} almost provides network operator a complete new business market with a huge amount of new end-users.
\par As an end-user aspect, \gls{webrtc} provides a much simpler way to have real-time conversation with another end-user. It is based on browser and internet which almost personal or enterprise computer already have, without any installation and plugins, end-user can have exactly the same service which previous stand-alone desktop client provides. By the prototype system of this thesis will cover, the end-user can even have the real-time rich communication service with multiple kinds of end-users.
\par This thesis will cover the research about how to apply \gls{webrtc} technology with existing legacy \gls{voip} network.

\noindent \textbf{Keywords} : \keywordnames
\end{abstract}